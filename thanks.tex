% Lời cam đoan
\begin{center}
    \Large{\textbf{LỜI CAM ĐOAN}}
\end{center}
\vspace{10mm}
\par
Tôi xin cam đoan đây là công trình nghiên cứu đề tài của riêng tôi. Các tài liệu, kết luận được sử dụng trong luận văn có nguồn gốc rõ ràng, đã công bố theo đúng quy định. Đây là đề tài do chính tôi thực hiện và chưa từng sử dụng để nộp lấy bằng cấp ở những nơi khác.
\vspace{1em}
\par
Tôi xin hoàn toàn chịu trách nhiệm về lời cam đoan này.
\vspace{10mm}
\begin{table}[!ht]
    \raggedleft
    \begin{tabular}{c}
        TP. Hồ Chí Minh, Tháng 06/2019\\
        Sinh viên thực hiện\\
        \vspace{5mm}\\
        Nguyễn Phước Thịnh\\
    \end{tabular}
\end{table}
% Lời cảm ơn
\thispagestyle{empty}
\cleardoublepage
\begin{center}
    \Large{\textbf{LỜI CẢM ƠN}}
\end{center}
\vspace{10mm}
\par
Trân trọng gửi lời cảm ơn đến Khoa Khoa học \& Kỹ thuật Máy tính, trường Đại học Bách Khoa TP. Hồ Chí Minh đã tạo điều kiện thuận lợi để tôi có thể học tập tốt và hoàn thành đề tài luận văn này.
\vspace{1em}
\par
Đặc biệt, xin bày tỏ lòng biết biết ơn sâu sắc đến thầy Nguyễn Đức Thái đã hướng dẫn chỉ dạy tôi trong quá trình thực hiện đề tài. Bên cạnh đó, thầy Nguyễn Hồng Nam đã tận tình giúp tôi khắc phục những thiếu sót để tôi có thể hoàn thiện bài luận văn của mình.
\vspace{1em}
\par
Tôi cũng không quên cảm ơn đến gia đình, bạn bè xung quanh đã ủng hộ và giúp tôi có động lực trong suốt thời gian thực hiện đề tài.
\vspace{1em}
\par
Trong quá trình thực hiện luận văn, mặc dù đã cố gắng hoàn thiện đề tài, trao đổi và tiếp thu ý kiến từ các thầy hướng dẫn, tuy nhiên chắc chắn không tránh khỏi những sai sót. Vì vậy tôi mong nhận được sự thông cảm và góp ý chân thành từ mọi người để tôi có thể hoàn thiện trong những đề tài tiếp theo.
\vspace{1em}
\par
Xin chân thành cảm ơn!
\vspace{10mm}
\begin{table}[!ht]
    \raggedleft
    \begin{tabular}{c}
        TP. Hồ Chí Minh, Tháng 06/2019\\
        Sinh viên thực hiện\\
        \vspace{5mm}\\
        Nguyễn Phước Thịnh\\
    \end{tabular}
\end{table}
\thispagestyle{empty}
\cleardoublepage
% Tóm tắt
\begin{center}
    \Large{\textbf{TÓM TẮT}}
\end{center}
\vspace{10mm}
\par
Với sự phát triển không ngừng của Internet và độ phổ biến của nó trên toàn thế giới, việc khai thác nội dung số để quảng bá website của mình đến với người dùng là hết sức quan trọng. Tuy nhiên, làm thế nào để có thể tiếp cận với số đông người dùng là việc làm không hề đơn giản. Mục tiêu là tăng khả năng xuất hiện khi người dùng tìm kiếm nội dung liên quan đến website của mình, gồm tập hợp các phương pháp nhằm cải thiện thứ hạng website trên trang kết quả tìm kiếm, hay còn gọi là tối ưu hóa công cụ tìm kiếm (Search Engine Optimization - SEO).
\vspace{1em}
\par
Do đó, chúng tôi quyết định nghiên cứu về đề tài này và xây dựng nên công cụ tự động phân tích mã nguồn website theo các tiêu chuẩn SEO hiện có, sau đó hiển thị kết quả đánh giá giúp người dùng có thể tối ưu website của họ.
\vspace{1em}
\par
Những nội dung chính sẽ được trình bày trong luận văn:
\begin{itemize}
    \item Chương I: Giới thiệu đề tài
    \item Chương II: Nền tảng lý thuyết
    \item Chương III: Các tiêu chuẩn SEO
    \item Chương IV: Thiết kế giải pháp
    \item Chương V: Hiện thực
    \item Chương VI: Kết luận đánh giá
\end{itemize}
\thispagestyle{empty}
\cleardoublepage