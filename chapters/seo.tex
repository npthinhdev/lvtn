\chapter{Các tiêu chuẩn SEO}
Để thu hút khách hàng truy cập và sử dụng webstie của mình, thì phải cần rất nhiều tiêu chí như giao diện đẹp, đáp ứng nhanh, thuận tiện mang đến trải nghiệm tốt cho người dùng. Tuy nhiên, ứng dụng của chúng tôi không đánh giá về mặt người dùng mà nó được dùng tối ưu để triển khai cho các công cụ tìm kiếm, giúp cho trang web được thứ hạng tốt hơn để người dùng có thể dễ dàng nhìn thấy, còn việc giữ chân khách hàng và sử dụng dịch vụ thì nó không là nhiệm vụ hàng đầu ứng dụng của chúng tôi.
\par
Sau đây là các tiêu chuẩn cần thiết cho 1 website chuẩn SEO\cite{seo}:
\section{Tên miền và hosting}
Khi lựa chọn tên miền và hosting, tốt nhất nên lựa chọn tên liên quan đến nội dung website hoặc liên quan đến ngành nghề của cửa hàng hay doanh nghiệp. Tên miền được sử dụng lâu sẽ được đánh giá cao, bên cạnh đó nên chọn tên miền ngắn gọn, dễ nhớ. Song song với tên miền thì hosting là yếu tố đi kèm quan trọng không thể thiếu, hosting có tốc độ nhanh, hoạt động ổn định với băng thông và dung lượng thích hợp, ít khi bị gián đoạn là một trong những yếu tố giúp website thân thiện hơn với công cụ tìm kiếm.
\section{Tốc độ tải trang}
Tốc độ tải trang của một website  là một yếu tố quan trọng trong việc thiết kế một trang web chuẩn SEO, vì vậy việc tối ưu cấu trúc code web giúp việc tải trang nhanh hơn. Thông thường một website được đánh giá cao khi có tốc độ tải trang trung bình từ 0.5 giây đến 2 giây. Một website có tốc độ tải trang chậm có thể sẽ làm cho người dùng cảm thấy khó chịu, đồng thời việc xếp hạng trên bảng công cộng tìm kiếm của Google cũng sẽ gặp khó khăn vì vậy cần phải chú ý vào tốc độ tải của một trang web khi thiết kế.
\section{URL website}
Bên cạnh tốc độ tải trang thì việc tối ưu URL website là điều quan trọng không kém, địa chỉ URL cần xuất hiện một cách rõ ràng và thân thiện ngay trên các thanh công cụ tìm kiếm. Khi thiết kế website chuẩn SEO thì phải xây dựng một URL có khả năng tùy biến và thân thiện.
\par
Một số yếu tố cần thiết khi xây dựng một URL tốt:
\begin{itemize}
	\item URL nên mã hóa theo tiêu đề bài viết, có liên quan tới nội dung bài viết, keywords, description.
	\item Dễ dàng thêm canonical URL cho những trang có nội dung bị trùng lặp nhiều.
	\item Sử dụng rewrite URL đối với các liên kết và phân tách mỗi từ bằng dấu gạch nối.
	\item Chỉ có 1 trong 2 có đuôi \texttt{www} hoặc không có \texttt{www} trước URL để tránh bị phạt nội dung trùng lặp.
	\item Không dùng các loại URL tự động có số bên trong như \url{www.example.com/?p=56789}.
	\item Có 1 file \texttt{robots.txt} chuẩn để điều khiển các crawler 1 cách hợp lý.
\end{itemize}
\section{Thẻ tiêu đề và mô tả}
Thẻ tiêu đề (title) và thẻ mô tả (meta description) được đánh giá là hai thẻ quan trọng nhất đối với một website.
\begin{itemize}
	\item Thẻ title cần đảm bảo ngắn gọn, xúc tích, duy nhất và không nên dài quá 65 ký tự.
	\item Thẻ meta description phải tóm tắt được nội dung chính trên website, hay một bài viết một cách ngắn gọn, đầy đủ thông tin và đặt biệt là không được dài quá 150 ký tự.
\end{itemize}
\section{Sitemap và robots.txt}
Khi thiết kế website chuẩn SEO không thể bỏ qua sơ đồ trang (sitemap) giúp phân chia cấu trúc website và cấu trúc URL khi duyệt website tốt hơn. Ngoài sitemap, website còn cần có thêm file \texttt{robots.txt} giúp crawler của các bộ máy tìm kiếm có thể thu thập thông tin từ website một cách đầy đủ, nhanh chóng và thuận thiện hơn.
\section{Thiết bị di động}
Website thân thiện với thiết bị di động, có thể tương tác và hiển thị nội dung tốt, trên mọi trình duyệt hay các thiết bị là một tiêu chuẩn mới cũng có thể xem đó là điểm cộng với một website chuẩn SEO. Ngày nay, người dùng có xu hướng sử dụng nhiều thiết bị khác nhau để tra cứu thông tin như PC, máy tính bảng hay điện thoại di động, do đó mà việc xây dựng một giao diện phù hợp và thích ứng được với nhiều thiết bị khác nhau là điều mà một website cần có.
\section{Các yếu tố khác}
Bên cạnh các yếu tố quan trọng đã được nêu thì còn có một số yếu tố cơ bản cân thiết khác khi xây dựng một website chuẩn SEO.
\begin{itemize}
	\item Website có giao diện độc đáo, bố cục phù hợp với cấu trúc thuận tiện và logic, chuẩn UX/UI giúp người dùng có thể tìm đến bất kỳ nội dung nào trên website một cách thuận tiện.
	\item Tối ưu các thẻ H1 đến H6 trên website cùng với thuận tiện trong việc tối ưu thẻ mô tả (thuộc tính alt) của hình ảnh.
	\item Tối ưu hiển thị cho bài viết, hình ảnh, video và các nội dung khác trên website.
	\item Nội dung được tải về trực tiếp trong code HTML, không phải thông qua JavaScript, AJAX, iframe\ldots
	\item Tích hợp thêm các nút chia sẻ mạng xã hội như nút Like Faceook, nút G+1 hay nút Tweet giúp việc chia sẻ nội dung website trên cộng đồng mạng trở nên nhanh chóng và thuận tiện hơn.
	\item Tối ưu cấu trúc website theo chuẩn W3C (Word Wide Web Consortium) đồng thời thân thiện với trải nghiệm người dùng.
	\item Đảm bảo tính bảo mật của website, hạn chế tối đa các hoạt động tấn công mạng hay virus tấn công website.
	\item Tạo trang 404 cho các trang không tìm thấy.
\end{itemize}