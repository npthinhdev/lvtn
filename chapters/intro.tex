\chapter{Giới thiệu đề tài}
\section{Tính cấp thiết}
SEO viết tắt của Search Engine Optimization có nghĩa là tối ưu hóa công cụ tìm kiếm, bao gồm tập hợp các phương pháp nhằn cải thiện thứ hạng website trang trang kết quả tìm kiếm, nổi trội nhất là trang \href{https://www.google.com}{Google}.
\par
Có rất nhiều tiêu chí để đánh giá một website có chuẩn SEO hay không như mức độ phân bổ từ khóa, số lượng liên kết bên ngoại, tốc độ tải trang,\ldots Trong đó, cấu trúc mã nguồn website là một trong những yếu tố quan trọng giúp các trình thu thập dữ liệu của công cụ tìm kiếm phân tích website.
\par
Hiện nay đã có nhiều website cung cấp dịch vụ đánh giá SEO, tuy nhiên hầu hết chúng đều thu phí người dùng và chưa hỗ trợ tiếng Việt. Do đó chúng tôi tạo ra ứng dụng này để đáp ứng nhu cầu sử dụng của người dùng Việt Nam và hoàn toàn miễn phí. Người dùng sẽ không cần phải am hiểu nhiều về lập trình mà vẫn có thể dễ dàng hiểu và sử dụng được ứng dụng của chúng tôi.
\par
Chúng tôi sử dụng Python để tạo nên ứng dụng này, vì Python trong những năm gần đây được quan tâm và phát triển vượt bậc. Với hàng ngàn thư viện được chia sẻ miễn phí, việc sử dụng Python sẽ giúp ứng dụng của chúng tôi có thể mở rộng nhiều hơn trong tương lai.
\section{Mục tiêu}
Ứng dụng của chúng tôi sẽ đặt ra những mục tiêu để hoàn thành sau đây:
\begin{itemize}
	\item Cung cấp dịch vụ đánh giá miễn phí cho người sử dụng.
	\item Tự động phân tích mã nguồn website bằng liên kết người dùng nhập vào.
	\item Chấm điểm website dựa trên phân tích cấu trúc SEO và mô tả kết quả.
	\item Cung cấp những tiêu chuẩn SEO đến người dùng thông qua các bài viết trên website.
\end{itemize}
\section{Phương pháp thực hiện}
Chúng tôi sử dụng Python để lập trình backend cho ứng dụng của mình. Python được biết đến là ngôn ngữ dành cho tính toán và phân tích nên thích hợp để xử lý các cú pháp cho website chúng tôi cần kiểm tra.
\par
Bên cạnh đó, với kho thư viện đồ sộ được chia sẻ công khai và miễn phí trên \url{https://pypi.org} sẽ giúp chúng tôi triển khai nhanh chóng ứng dụng nhờ vào các framework mở được chia sẻ để sử dụng.
\par
Những framework và thư viện chúng tôi sẽ sử dụng cho ứng dụng của mình:
\begin{itemize}
	\item Django: Framework Python dùng để phát triển ứng dụng web.
	\item Requests, Lxml: Thư viện Python có nhiệm vụ phân tích cú pháp và lấy từng phần tử của website chúng tôi cần kiểm tra.
	\item reCAPTCHA: Ứng dụng do Google phát triển, nhằm hạn chế spam và BOT tác động lên website để giữ trang web trở nên an toàn.
	\item Bootstrap: Thư viện dùng để thiết kế giao diện cho trang web, nó hỗ trợ tốt cho việc hiển thị website đa nền tảng.
	\item Font Awesome: Thư viện cung cấp các icon cần thiết cho giao diện website.
	\item Heroku: Cloud platform miễn phí để chúng tôi triển khai ứng dụng Python của mình lên Internet.
\end{itemize}
\section{Bố cục báo cáo}
Tiếp theo, những phần sau chúng tôi sẽ trình bày cách chúng tôi sử dụng những framework và thư viện để xây dựng nên website cùng với những tiêu chí được áp dụng để đánh giá SEO cho một trang web. Phần kết luận đánh giá sẽ được giới thiệu vào mục cuối cùng kèm theo những tài liệu tham khảo mà chúng tôi sử dụng để hoàn thành báo cáo này.