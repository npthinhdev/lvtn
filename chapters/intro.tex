\chapter{Giới thiệu đề tài}
\section{Tính cấp thiết}
Với sự lớn mạnh không ngừng của Internet và độ phổ biến của nó trên toàn Thế giới, việc các doanh nghiệp và cửa hành kinh doanh sử dụng nó để mang lại doanh thu cho mình là hết sức quan trọng. Tuy nhiên, có rất nhiều đơn vị muốn chiếm lĩnh vị trí cao trên nó dẫn đến việc cạnh tranh rất gay gắt trong công cuộc đưa website của mình đến với đại đa số khách hàng ở khắp mọi nơi, được gọi tắt là SEO (Search Engine Optimization).
\par
Từ đó việc tối ưu trang web để mang lại thứ hạng cao trên các công cụ tìm kiếm ra đời. Người lập trình viên ngoài việc phải đáp ứng các kịch bản của ứng dụng mà còn phải đảm bảo tối ưu website của họ đối với các công cụ tìm kiếm. Để giúp cho công việc của họ được thuận tiện hơn và tuân thủ được các tiêu chí tối ưu mới nhất, chúng tôi cung cấp một giải pháp tự động, phân tích cú pháp trang web và đưa ra những đề nghị sửa chữa, nhằm tối ưu SEO cho trang web của họ.
\section{Mục tiêu}
Để mang đến sự thuận tiện và hài lòng cho người sử dụng, ứng dụng của chúng tôi sẽ đặt ra những mục tiêu sau đây:
\begin{itemize}
	\item Cung cấp dịch vụ đánh giá miễn phí! Vì thế chúng tôi sẽ đặt quảng cáo lên trang web để mang lại nguồn thu duy trì cho chúng tôi.
	\item Cập nhật những tiêu chuẩn về SEO đến người dùng thông qua các bài viết trên website và đảm bảo rằng công cụ của chúng tôi sẽ sử dụng những tiêu chuẩn mới nhất.
	\item Tự động phân tích cú pháp trang web người dùng, so sánh với các tiêu chuẩn về SEO, đưa ra kết quả, đánh giá và gợi ý cho người dùng cách để họ có thể khắc phục nếu website chưa đạt chuẩn hoặc thiếu các thành phần quan trọng.
\end{itemize}
\section{Phương pháp thực hiện}
Chúng tôi sử dụng Python để làm ngôn ngữ lập trình cốt lõi cho ứng dụng của mình. Python được biết đến là ngôn ngữ dành cho tính toán và phân tích nên sẽ thích hợp để xử lý các cú pháp cho trang web chúng tôi cần kiểm tra.
\par
Bên cạnh đó, với kho thư viện đồ sộ được chia sẻ công khai trên \url{https://pypi.org} sẽ giúp chúng tôi triển khai nhanh chóng ứng dụng nhờ vào các framework mở được chia sẻ miễn phí để sử dụng.
\par
Cụ thể hơn, sau đây là những framework và thư viện chúng tôi sẽ sử dụng cho ứng dụng của mình:
\begin{itemize}
	\item Django: Framework Python dùng để phát triển ứng dụng web.
	\item Requests, Lxml: Thư viện Python có nhiệm vụ phân tích cú pháp và lấy từng phần tử của website chúng tôi cần kiểm tra.
	\item reCAPTCHA: Ứng dụng do Google phát triển, nhằm hạn chế spam và BOT tác động lên website để giữ trang web trở nên an toàn.
	\item Bootstrap: Thư viện dùng để thiết kế giao diện cho trang web của chúng tôi, nó hỗ trợ tốt cho việc hiển thị website đa nền tảng.
	\item Font Awesome: Thư viện cung cấp các icon cần thiết cho giao diện website.
	\item Heroku: Cloud platform miễn phí để chúng tôi triển khai ứng dụng Python của mình lên Internet.
\end{itemize}
\section{Bố cục báo cáo}
Tiếp theo, những phần sau chúng tôi sẽ trình bày cách chúng tôi sử dụng những framework và thư viện để xây dựng nên website cùng với những tiêu chí được áp dụng để đánh giá SEO cho một trang web. Phần kết luận đánh giá sẽ được giới thiệu vào mục cuối cùng kèm theo những tài liệu tham khảo mà chúng tôi sử dụng để hoàn thành báo cáo này.