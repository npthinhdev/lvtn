\chapter{Nền tảng lý thuyết}
Phần này chúng tôi sẽ giới thiệu cũng như trang bị những kiến thức cơ bản để sử dụng những công cụ được liệt kê ở phần trước.
\section{Ngôn ngữ Python}
Python hiện đang là một trong những ngôn ngữ lập trình phổ biến. Một phần nhờ vào khả năng dễ tiếp cận, cấu trúc rõ ràng và quan trọng hơn, nó có thể giải quyết tốt các bài toán kỹ thuật với thời gian thực thi nhanh và tiết kiệm dòng code. Python được tạo ra bởi Guido van Rossum và phát hành vào năng 1991.\cite{python}
\par
Phiên bản sử dụng: 3.7.3
\begin{itemize}
	\item Biến (Variable): Không giống với các ngôn ngữ khác, Python không có câu lệnh riêng biệt để khai báo biến. Biến không cần phải khai báo kiểu giá trị nào và có thể thay đổi dựa vào giá trị mà nó được gán.
	\begin{lstlisting}[language=Python]
# x is of type int
x = 5
# x is now of type str
x = "Thinh"
	\end{lstlisting}
	\item Chuỗi (String): Chuỗi ký tự trong Python được chứa trong cặp dấu nháy đơn hoặc dấu nháy kép. Để hiển thị chuỗi ra màn hình, sử dụng lệnh \texttt{print()}.
	\begin{lstlisting}[language=Python]
a = "Hello, World!"
print(a)

>>>"Hello, World"
	\end{lstlisting}
	\item Toán tử (Operator):
	\begin{itemize}
		\item Số học:
		\begin{table}[!ht]
			\centering
			\begin{tabular}{|c|l|}
				\hline
				$+$ & Cộng\\
				\hline
				$-$ & Trừ\\
				\hline
				$*$ & Nhân\\
				\hline
				$/$ & Chia\\
				\hline
				$\%$ & Chia lấy phần dư\\
				\hline
				$**$ & Lũy thừa\\
				\hline
				$//$ & Chia lấy phần nguyên\\
				\hline
			\end{tabular}
			\caption{Toán tử Số học}
		\end{table}
		\item So sánh:
		\begin{table}[!ht]
			\centering
			\begin{tabular}{|c|l|}
				\hline
				$==$ & Bằng\\
				\hline
				$!=$ & Không bằng\\
				\hline
				$>$ & Lớn hơn\\
				\hline
				$<$ & Nhỏ hơn\\
				\hline
				$>=$ & Lớn hơn hoặc bằng\\
				\hline
				$<=$ & Nhỏ hơn hoặc bằng\\
				\hline
			\end{tabular}
			\caption{Toán tử So sánh}
		\end{table}
		\item Logic:
		\begin{table}[!ht]
			\centering
			\begin{tabular}{|c|l|}
				\hline
				$and$ & Trả về \texttt{True} nếu 2 điều kiện đều đúng\\
				\hline
				$or$ & Trả về \texttt{True} nếu 1 trong 2 điều kiện là đúng\\
				\hline
				$not$ & Đảo ngược kết quả của điều kiện\\
				\hline
			\end{tabular}
			\caption{Toán tử Logic}
		\end{table}
		\item Identity:
		\begin{table}[!ht]
			\centering
			\begin{tabular}{|c|l|}
				\hline
				$is$ & Trả về \texttt{True} nếu 2 biến cùng trỏ tới 1 đối tượng\\
				\hline
				$is$ $not$ & Trả về \texttt{True} nếu 2 biến không trỏ cùng đối tượng\\
				\hline
			\end{tabular}
			\caption{Toán tử Identity}
		\end{table}
		\item Membership:
		\begin{table}[!ht]
			\centering
			\begin{tabular}{|c|l|}
				\hline
				$in$ & Trả về \texttt{True} nếu biến nằm trong tập hợp các biến\\
				\hline
				$not$ $in$ & Trả về \texttt{True} nếu biến không nằm trong tập hợp các biến\\
				\hline
			\end{tabular}
			\caption{Toán tử Membership}
		\end{table}
	\end{itemize}
	\item Dictionary: Tập hợp không có thứ tự, có thể thay đổi và lập chỉ mục. Được biểu diễn bằng cặp dấu ngoặc nhọn, bên trong là khóa (key) và giá trị (value) tương ứng.
	\begin{lstlisting}[language=Python]
hoten = {
	"ho": "Nguyen Phuoc",
	"ten": "Thinh"
}
	\end{lstlisting}
	\item Câu điều kiện (If\ldots Else): Dùng để thực thi một hành động sau khi thỏa điều kiện cho trước. Lưu ý trong Python, sử dụng thụt lề dòng để phân biệt các khối lệnh với nhau.
	\begin{lstlisting}[language=Python]
a = 1
b = 2
if a > b:
	print("a is greater than b")
elif a == b:
	print("a and b are equal")
else:
	print("b is greater than a")

>>>"b is greater than a"
	\end{lstlisting}
	\item Vòng lặp (For): Dùng để lặp qua một chuỗi (có thể là list, tuple, dictionary, set hoặc string).
	\begin{lstlisting}[language=Python]
hoten = ["Nguyen", "Phuoc", "Thinh"]
for x  in hoten:
	print(x)
	
>>>"Nguyen"
>>>"Phuoc"
>>>"Thinh"
	\end{lstlisting}
	\item Hàm (Function): Gồm một khối code, được khởi chạy khi được gọi đến. Để truyền dữ liệu vào 1 hàm được gọi là tham số (parameter). Hàm trả về kết quả thông qua lệnh \texttt{return}.
	\begin{lstlisting}[language=Python]
# a function is defined using the def keyword
def add(n):
	return 1 + n
# calling a function
add(1)

>>>2
	\end{lstlisting}
	\item Lớp/Đối tượng (Class/Object): Python là ngôn ngữ lập trình hướng đối tượng. Hầu hết mọi thứ trong Python là một đối tượng (object), gồm thuộc tính và phương thức của nó. Sử dụng lớp (class) để khởi tạo 1 đối tượng mới.
	\begin{lstlisting}[language=Python]
# create a class
class Person:
	def __init__(self, name, age):
		self.name = name
		self.age = age
	# object method
	def func(self):
		print(f"My name is: {self.name}, {self.age} years old")
# create object
p = Person("Thinh", 20)
p.func()
# modify object property
p.age = 22
print(p.age)

>>>"My name is: Thinh, 20 years old"
>>>22
	\end{lstlisting}
	\item Module: Có thể xem module là 1 bộ thư viện mã code, được lưu bởi tệp hoặc thư mục tách biệt với project đang thực thi, được nhúng vào để tái sử dụng những bộ code chứa trong đó.
	\par
	Để sử dụng các hàm hoặc lớp trong file \texttt{mymodule.py}, ta sử dụng lệnh sau:
	\begin{lstlisting}[language=Python]
import mymodule
# import only part from a module
from mymodule import myfunc
	\end{lstlisting}
	\item PIP: Là trình quản lý gói (package) hoặc module dành cho Python. Gói là nơi chứa tất cả các file cần thiết cho 1 module.
	\par
	Để cài đặt 1 gói trong Python, sử dụng lệnh sau:
	\begin{lstlisting}[language=bash]
\>pip install Django
	\end{lstlisting}
	\item Xử lý ngoại lệ (Try\ldots Except): Khi chương trình xảy ra lỗi hoặc ngoại lệ, Python sẽ dừng lại và đưa ra thông báo lỗi cho người dùng. Để tránh ứng dụng bị gián đoạn, sử dụng câu lệnh \texttt{try} để bắt và xử lý các ngoại lệ khi chương trình đang được thực thi.
	\begin{lstlisting}[language=Python]
try:
	print(x)
except NameError:
	print("Variable x is not defined")
	
>>>"Variable x is not defined"
	\end{lstlisting}
\end{itemize}
\section{Framework Django}
Django là 1 framework Python web cấp cao, thúc đẩy phát triển nhanh chóng, gọn gàng và tiện dụng. Được xây dựng bởi các nhà lập trình viên có kinh nghiệm, xử lý được các vấn đề rắc rối khi phát triển web, do đó người dùng chỉ cần quan tâm hoàn thiện các chức năng cho web mà không cần phải quá lo lắng về nền tảng phía sau. Và quan trọng nó là mã nguồn mở và miễn phí.
\par
Chúng tôi sẽ sử dụng mục tài liệu\cite{django} tại trang chủ của Django framework để trình bày những khái niệm và cách để hiện thực ứng dụng của chúng tôi.
\par
Phiên bản sử dụng: 2.1.4
\begin{itemize}
	\item Sau khi cài đặt xong Django từ trình quản lý gói PIP của Python, dùng lệnh sau để khởi tạo project mới có tên là \texttt{lvtn}:
	\begin{lstlisting}[language=bash]
\>django-admin startproject lvtn
	\end{lstlisting}
	Một folder mới được tạo ra chứa các thành phần của project, có cấu trúc và chức năng như sau:
	\begin{lstlisting}[language=bash]
lvtn\
	lvtn\
		__init__.py
		settings.py
		urls.py
		wsgi.py
	manage.py
	\end{lstlisting}
	\begin{itemize}
		\item \texttt{manage.py}: Một CLI giúp tương tác với ứng dụng web.
		\item \texttt{lvtn\textbackslash\_\_init\_\_.py}: File rỗng, để chỉ cho Python biết thư mục này nên được xem là một gói.
		\item \texttt{lvtn\textbackslash settings.py}: Chứa các tùy chỉnh của project.
		\item \texttt{lvtn\textbackslash urls.py}: Các khai báo URL cho trang web.
		\item \texttt{lvtn\textbackslash wsgi.py}: Được sử dụng khi deploy project lên Internet.
	\end{itemize}
	\item Server phát triển: Dùng để khởi chạy ứng dụng web trên máy tính local.
	\begin{lstlisting}[language=bash]
\>python manage.py runserver
	\end{lstlisting}
	Khi server đang chạy, truy cập vào địa chỉ \url{http://127.0.0.1:8000} trên trình duyệt web để thấy ứng dụng đang được trình diễn.
	\item Tạo app mới: Mỗi ứng dụng được viết trong Django bao gồm 1 gói Python tuân thủ theo 1 quy ước nhất định. Django đi kèm với 1 tiện ích tự động tạo cấu trúc thư mục cơ bản của 1 ứng dụng, do đó người lập trình chỉ cần quan tâm đến việc phát triển code bên trong mà thôi.
	\par
	Tạo app \texttt{checkweb}	có nhiệm vụ xử lý chính cho project của chúng tôi.
	\begin{lstlisting}[language=bash]
\>python manage.py startapp checkweb
	\end{lstlisting}
	Thư mục mới được tạo ra có cấu trúc như sau:
	\begin{lstlisting}[language=bash]
checkweb\
	migrations\
		__init__.py
	__init__.py
	admin.py
	apps.py
	models.py
	tests.py
	views.py
	\end{lstlisting}
	\begin{itemize}
		\item \texttt{migrations\textbackslash}: Thư mục chứa các file được sinh ra khi có thay đổi về cấu trúc cơ sở dữ liệu.
		\item \texttt{admin.py}: Dùng để thiết đặt các thuộc tính được hiển trị trong trang quản trị admin mà Django cung cấp sẵn.
		\item \texttt{apps.py}: Khai báo app được sử dụng trong project, đảm bảo rằng các app không bị trùng lặp trong 1 dự án.
		\item \texttt{models.py}: Django hỗ trợ các phương thức để xử lý cơ sở dữ liệu mà không cần sử dụng đến các câu lệnh truy vấn SQL trực tiếp.
		\item \texttt{tests.py}: Được người dùng sử dụng để triển khai các kịch bản thử nghiệm và rà soát lỗi trước khi phát hành ứng dụng.
		\item \texttt{views.py}: Đóng vai trò xử lý trung tâm của ứng dụng, quản lý việc hiển thị, kết nối đến cơ sở dữ liệu và thực thi các hàm do lập trình viên thêm vào ứng dụng.
		\par
		Sau khi tạo xong app, cần phải khai báo trong project bằng các thêm dòng sau vào file \texttt{lvtn\textbackslash settings.py}:
		\begin{lstlisting}[language=Python]
INSTALLED_APPS = [
    "django.contrib.admin",
    "django.contrib.auth",
    "django.contrib.contenttypes",
    "django.contrib.sessions",
    "django.contrib.messages",
    "django.contrib.staticfiles",
    # add code below
    "checkweb.apps.CheckwebConfig",
]
	\end{lstlisting}
	\end{itemize}
	\item Migration: Khi tạo mới hoặc có sự thay đổi về cấu trúc cơ sở dữ liệu, migration có nhiệm vụ lưu lại quá trình ứng dụng thay đổi, do đó có thể truy vết lại và phục hồi lại những cập nhật trước đó 1 cách dễ dàng, nhất là khi ứng dụng gặp lỗi. Để thực hiện, dùng các lệnh sau:
	\begin{lstlisting}[language=bash]
\>python manage.py migrate
\>python manage.py makemigrations
\>python manage.py migrate
	\end{lstlisting}
	\item Trang quản trị admin: Một trong những ưu điểm của Django so với các framework khác là nó cung cấp trang Django administration giúp hiển trị trực quan cơ sở dữ liệu do người dùng thiết lập, cho phép xem, tạo mới, chỉnh sửa, xóa và nhiều tính năng khác nữa.
	\par
	Để có thể truy cập vào trang quản trị thì trước tiên cần phải tạo tài khoản \texttt{superuser}:
	\begin{lstlisting}[language=bash]
\>python manage.py createsuperuser
\>Username: ___
\>Email address: ___
\>Password: ___
\>Password (again): ___
\>Superuser created successfully.
	\end{lstlisting}
	Sau khi điền đầy đủ các thông tin trên thì có thể đăng nhập tài khoản để truy cập vào trang quản trị tại địa chỉ: \url{http://127.0.0.1:8000/admin/}
	\item Class-based view: Đây là chức năng được Django hỗ trợ, giúp lập trình viên ít phải viết code hơn để hiện thị 1 giao diện lên trình duyệt web. Nó hỗ trợ tốt trong việc truyền tham số, lấy giá trị từ model và có thể dễ dàng tùy chỉnh theo ý muốn.
	\par
	Để sử dụng, cần phải thêm module vào file muốn dùng nó. Đoạn code sau có chức năng hiển thị file \texttt{about.html} ra đường dẫn \url{http://127.0.0.1:8000/about/}.
	\begin{lstlisting}[language=Python]
from django.urls import path
from django.views.generic import TemplateView

urlpatterns = [
	path("about/", TemplateView.as_view(template_name="about.html")),
]
	\end{lstlisting}
	\item Django template: Dựa trên file \texttt{.html} nhưng có chèn thêm các đoạn code riêng biệt để mỗi khi chạy chương trình, Django sẽ render ra giao diện lên trình duyệt tương ứng.
	\begin{lstlisting}[language=HTML]

{{ section.title }}

<h1>{{ section.title }}</h1>

<h2>
 	<a href="{{ story.get_absolute_url }}">
		{{ story.headline|upper }}
	</a>
</h2>
<p>{{ story.tease|truncatewords:100 }}</p>


	\end{lstlisting}
\end{itemize}
\section{Thư viện Python}
\subsection{Requests}
Tham khảo\cite{requests}.
\par
Phiên bản sử dụng: 2.21.0
\begin{itemize}
	\item Cách cài đặt:
	\begin{lstlisting}[language=bash]
\>pip install requests
	\end{lstlisting}
	\item Phương thức \texttt{get}: Dùng để 	lấy toàn bộ nội dung trang web dựa trên tham số url.
	\begin{lstlisting}[language=Python]
import requests
page = requests.get(url)
	\end{lstlisting}
\end{itemize}
\subsection{Lxml}
Tham khảo\cite{lxml}.
\par
Phiên bản sử dụng: 4.3.3
\begin{itemize}
	\item Cách cài đặt:
	\begin{lstlisting}[language=bash]
\>pip install lxml
	\end{lstlisting}
	\item Gói \texttt{lxml.html}: Dùng để phân tách chuỗi HTML.
	\begin{lstlisting}[language=Python]
from lxml import html
content = html.fromstring(page.content)
value = content.xpath("//title/text()")
	\end{lstlisting}
\end{itemize}
\section{Thư viện giao diện}
\subsection{Bootstrap}
Tham khảo\cite{bootstrap}.
\par
Phiên bản sử dụng: 4.3.1
\begin{itemize}
 	\item CSS: Sao chép và dán dòng code bên dưới vào trong thẻ \texttt{<head>} trước tất cả các định dạng khác để tải CSS của Bootstrap.
 	\begin{lstlisting}[language=HTML]
<link rel="stylesheet" href="https://stackpath.bootstrapcdn.com/bootstrap/
4.1.3/css/bootstrap.min.css">
	\end{lstlisting}
	\item JS: Đặt trong thẻ \texttt{<script>} ở gần cuối trang web, trước khi đóng thẻ \texttt{</body>} để kích hoạt chúng. jQuery phải được đặt trước, đến Popper.js và sau cùng là phần JavaScript.
	\begin{lstlisting}[language=HTML]
<script src="https://code.jquery.com/jquery-3.3.1.slim.min.js"></script>
<script src="https://cdnjs.cloudflare.com/ajax/libs/popper.js/1.14.3/umd/
popper.min.js"></script>
<script src="https://stackpath.bootstrapcdn.com/bootstrap/4.1.3/js/
bootstrap.min.js"></script>
	\end{lstlisting}
\end{itemize}
\subsection{Font Awesome}
Tham khảo\cite{awesome}.
\par
Phiên bản sử dụng: 5.8.2
\par
Trước khi sử dụng được, cần phải chèn dòng code bên dưới vào thẻ \texttt{<head>} nằm ở đầu trang web.
\begin{lstlisting}[language=HTML]
<link rel="stylesheet" href="https://use.fontawesome.com/releases/v5.5.0/css/
all.css">
\end{lstlisting}
\par
Để chèn icon vào trang web, sử dụng dòng code tương tự cú pháp bên dưới:
\begin{lstlisting}[language=HTML]
<i class="fas fa-heart"></i>
\end{lstlisting}
\section{Bảo mật}
\subsection{reCAPTCHA}
Tham khảo\cite{captcha}.
\par
Phiên bản sử dụng: reCAPTCHA v2
\par
Cách đơn giản để sử dụng reCAPTCHA vào trang web bằng cách nhúng mã JavaScript và thẻ \texttt{g-recaptcha}. Thẻ \texttt{g-recaptcha} là 1 thẻ \texttt{DIV} với tên class là \texttt{"g-recaptcha"}, có thuộc tính \texttt{data-sitekey} chứa Site key được cấp khi đăng ký sử dụng reCAPTCHA.
\begin{lstlisting}[language=HTML]
<html>
	<head>
		<title>reCAPTCHA demo: Simple page</title>
		<script src="https://www.google.com/recaptcha/api.js" async defer></script>
	</head>
	<body>
		<form action="?" method="POST">
			<div class="g-recaptcha" data-sitekey="your_site_key"></div>
			<br/>
			<input type="submit" value="Submit">
		</form>
	</body>
</html>
\end{lstlisting}
\par
Sau khi submit form sử dụng reCAPTCHA, cần phải gửi giá trị có tên là \texttt{g-recaptcha-\\response} bằng phương thức POST về máy chủ của Google để xác minh người dùng đã xác thực bằng reCAPTCHA tại địa chỉ \url{https://www.google.com/recaptcha/api/siteverify}.
\begin{itemize}
	\item secret (bắt buộc): Khóa Secret key được cấp đồng thời với Site key khi đăng ký sử dụng.
	\item response (bắt buộc): Kết quả trả về của thuộc tính có tên là \texttt{g-recaptcha-response}.
	\item remoteip: Địa chỉ IP của người dùng cuối.
\end{itemize}
\par
Tiếp theo, Google sẽ trả về kết quả kiểm tra có thỏa reCAPTCHA dưới dạng JSON bằng các giá trị sau:
\begin{lstlisting}[language=Python]
{
	"success": true|false,
	"challenge_ts": timestamp, # timestamp of the challenge load
	"hostname": string, # the hostname of the site where the reCAPTCHA was solved
	"error-codes": [...] # optional
}
\end{lstlisting}
\par
Dựa vào kết quả này, chúng tôi có thể biết được người dùng đã có giải được captcha hay chưa, sau đó tiến hành các yêu cầu từ người dùng.	
\section{Triển khai ứng dụng}
\subsection{Heroku platform}
Tham khảo\cite{heroku}.
\par
Đầu tiên và quan trọng nhất, các ứng dụng Heroku yêu cầu 1 file \texttt{Procfile} để cài đặt nền tảng sử dụng, được đặt tại thư mục gốc.
\par
\texttt{Procfile}
\begin{lstlisting}[language=sh]
web: gunicorn myproject.wsgi
\end{lstlisting}
\par
File \texttt{Procfile} này yêu cầu \texttt{Gunicorn}, 1 máy chủ web được khuyến nghị dùng cho ứng dụng Django. Để cài đặt, sử dụng lệnh:
\begin{lstlisting}[language=bash]
\>pip install gunicorn
\end{lstlisting}
\par
Thay đổi trong file \texttt{settings.py} của ứng dụng Django. Khi sử dụng Heroku, các thông tin nhạy cảm sẽ được lưu trữ trong môi trường được gọi là config vars. Nó bao gồm các thông tin để kết nối đến cơ sở dữ liệu, trong khi bình thường sẽ được ghi trong file \texttt{settings.py} của Django.
\par
Gói \texttt{django-heroku} sẽ tự động cấu hình ứng dụng Django để nó hoạt động trên Heroku. Nó tương thích với các ứng dụng Django 2.0. Để cài đặt, sử dụng lệnh:
\begin{lstlisting}[language=bash]
\>pip install django-heroku
\end{lstlisting}
\par
Sau khi cài đặt, cần phải \texttt{import} câu lệnh sau vào đầu file \texttt{settings.py}:
\begin{lstlisting}[language=Python]
import django_heroku
\end{lstlisting}
\par
Sau đó thêm phần sau vào cuối file \texttt{settings.py}:
\begin{lstlisting}[language=Python]
# activate django-heroku.
django_heroku.settings(locals())
\end{lstlisting}
\par
Triển khai ứng dụng và hoàn tất.