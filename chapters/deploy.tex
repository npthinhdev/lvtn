\chapter{Hiện thực}
Phần này chúng tôi sẽ trình bày quá trình chúng tôi xây dựng và phát triển để tạo thành sản phẩm hoàn chỉnh. Dựa trên nền tảng lý thuyết được giới thiệu ở phần trước, chúng tôi sẽ áp dụng chúng vào những hướng dẫn bên dưới. Ở phần cuối của chương này sẽ là hướng dẫn về cách triển khai ứng dụng lên Internet bằng việc sử dụng máy chủ do trang Heroku cung cấp.
\par
Lưu ý, trong bài báo cáo này chúng tôi đang sử dụng môi trường lập trình là Windows nên có thể có những câu lệnh sẽ khác với MacOS, Linux hay những môi trường lập trình khác.
\section{Khởi tạo project}
\subsection{Tạo thư mục và thiết lập môi trường}
Chúng tôi quyết định đặt tên dự án của mình là \textbf{\texttt{check-seo}}, do đó chúng tôi sẽ tạo thư mục mới để lưu trữ code.
\par
Để tạo mới thư mục, bạn có thể click chuột phải $\rightarrow$ chọn New $\rightarrow$ chọn Folder $\rightarrow$ đặt tên \textbf{\texttt{check-seo}}.
\par
Ở đây, chúng tôi sẽ tạo thư mục bằng command line trong Windows PowerShell. Để mở PowerShell tại thư mục hiện tại, bấm giữ phím \textbf{\texttt{Shift}} $\rightarrow$ click chuột phải chọn Open PowerShell window here $\rightarrow$ nhập lệnh sau để tạo thư mục mới:
\begin{lstlisting}[language=bash]
\>mkdir check-seo
\end{lstlisting}
\par
Di chuyển khung làm việc vào thư mục project vừa tạo.
\begin{lstlisting}[language=bash]
\>cd check-seo
\end{lstlisting}
\par
Tại đây, chúng tôi sẽ tiến hành cài đặt môi trường để lập trình cho ứng dụng Python của mình. Đảm bảo là bạn đã cài đặt xong Python theo hướng dẫn ở phần \textbf{\textit{Nền tảng lý thuyết}}, để tránh dài dòng, chúng tôi sẽ hạn chế nhắc lại những kiến thức đã được trình bày ở phần trước.
\par
Tại khung cửa sổ của PowerShell, nhập lệnh sau để cài đặt môi trường:
\begin{lstlisting}[language=bash]
\>python -m venv ./venv
\end{lstlisting}
\par
Thư mục mới được tạo ra có tên là \textbf{\texttt{venv}} chứa các tập tin hệ thống giúp tạo môi trường ảo cho Python. Để kích hoạt môi trường ảo, sử dụng câu lệnh:
\begin{lstlisting}[language=bash]
\>.\venv\Scripts\activate
\end{lstlisting}
\par
Khi kích hoạt môi trường thành công, sẽ có phần thông tin \textbf{\texttt{(venv)}} hiển thị ở đầu mỗi dòng lệnh, giống như \textbf{\texttt{(venv)\textbackslash>}}
\subsection{Cài đặt các thư viện cần thiết}
Sau khi tạo xong thư mục và kích hoạt xong môi trường ảo, tiếp theo chúng tôi sẽ tiến hành cài đặt các thư viện phục vụ cho dự án của chúng tôi.
\par
Trong thư mục \textbf{\texttt{check-seo}}, tạo tệp mới có tên là \textbf{\texttt{requirements.txt}} sẽ chứa thông tin về thư viện và phiên bản sử dụng, để xem thông tin cụ thể của từng thư viện, bạn có thể tìm kiếm chúng trên kho lưu trữ công khai của Python là \url{https://pypi.org}. Nội dung của file \textbf{\texttt{requirements.txt}} như sau:
\begin{lstlisting}
Django==2.2
lxml==4.3.3
requests==2.21.0
\end{lstlisting}
\par
Để tiến hành cài đặt thư viện được liệt kê trong file \textbf{\texttt{requirements.txt}}, sử dụng câu lệnh:
\begin{lstlisting}[language=bash]
(venv)\>pip install -r requirements.txt
\end{lstlisting}
\par
Trình cài đặt thư viện Python sẽ tiến hành tải về và cài đặt trong môi trường ảo mà chúng tôi đã kích hoạt. Ngoài thư viện chính, trình cài đặt còn tải thêm những thư viện khác bổ trợ đi theo từng thư viện. Để kiểm tra các gói đã cài đặt, sử dụng lệnh:
\begin{lstlisting}[language=bash]
(venv)\>pip freeze list
\end{lstlisting}
\par
Kết quả trả về có thể như sau:
\begin{lstlisting}
certifi==2019.3.9
chardet==3.0.4
Django==2.2
idna==2.8
lxml==4.3.3
pytz==2019.1
requests==2.21.0
sqlparse==0.3.0
urllib3==1.24.3
\end{lstlisting}
\subsection{Tạo project và xây dựng app Django}
Sau khi cài đặt xong những thư viện cần thiết, tiếp theo, chúng ta sẽ tiến hành tạo mới project Django có tên là \textbf{\texttt{src}} trong thư mục \textbf{\texttt{check-seo}} bằng câu lệnh:
\begin{lstlisting}[language=bash]
(venv)\>django-admin startproject src .
\end{lstlisting}
\par
Sau khi thực thi thành công câu lệnh trên thì sẽ tạo ra thư mục \textbf{\texttt{src}} chứa các file cài đặt cho project và file \textbf{\texttt{manage.py}} giúp quản lý các thao tác command line cho ứng dụng.
\par
Theo thiết kế của ứng dụng, chúng tôi sẽ tạo thêm 2 app cho project là \textbf{\texttt{checkweb}} quản lý chính cho việc thu thập, đánh giá SEO cho website và \textbf{\texttt{tips}} sẽ đảm nhiệm hiển thị các bài đăng về thủ thuật SEO. Để tạo app, sử dụng lần lượt 2 câu lệnh sau đây:
\begin{lstlisting}[language=bash]
(venv)\>python .\manage.py startapp checkweb
(venv)\>python .\manage.py startapp tips
\end{lstlisting}
\par
Để quản lý các file giao diện \textbf{\texttt{html}}, chúng tôi tạo thêm thư mục \textbf{\texttt{templates}} tại thư mục chính của project. Tiếp theo, chúng tôi đi vào thư mục \textbf{\texttt{src}} sau đó tạo thêm thư mục có tên là \textbf{\texttt{static\_venv}} đảm nhiệm việc lưu trữ các file CSS, JavaScript và các thư viện bên ngoài như Bootstrap, jQuery.
\par
Sau khi tạo mới app, thư mục \textbf{\texttt{templates}} và \textbf{\texttt{static\_venv}}, cần phải đăng ký vào cấu hình để project hiểu được cấu trúc của chương trình tại file \textbf{\texttt{settings.py}} trong thư mục \textbf{\texttt{src}}.
\par
Để khai báo app, tìm đến dòng \textbf{\texttt{INSTALLED\_APPS}} và thêm đoạn code bên dưới vào hàng cuối cùng, kết quả sẽ tương tự như:
\begin{lstlisting}[language=Python]
INSTALLED_APPS = [
    "django.contrib.admin",
    "django.contrib.auth",
    "django.contrib.contenttypes",
    "django.contrib.sessions",
    "django.contrib.messages",
    "django.contrib.staticfiles",

    "checkweb.apps.CheckwebConfig",
    "tips.apps.TipsConfig",
]
\end{lstlisting}
\par
Cấu hình templates cho project tại khóa \textbf{\texttt{DIRS}} của mục \textbf{\texttt{TEMPLATES}}:
\begin{lstlisting}[language=Python]
TEMPLATES = [
    {
        "BACKEND": "django.template.backends.django.DjangoTemplates",
        "DIRS": [os.path.join(BASE_DIR, "templates")],
        "APP_DIRS": True,
        "OPTIONS": {
            "context_processors": [
                "django.template.context_processors.debug",
                "django.template.context_processors.request",
                "django.contrib.auth.context_processors.auth",
                "django.contrib.messages.context_processors.messages",
            ],
        },
    },
]
\end{lstlisting}
\par
Cuối cùng trong phần này, chúng tôi sẽ cấu hình phần \textbf{\texttt{static}} để hiển thị các file CSS, JavaScript,\ldots tại mục \textbf{\texttt{STATIC\_URL}}, chúng tôi sẽ thêm đoạn code vào để được kết quả như dưới đây:
\begin{lstlisting}[language=Python]
STATIC_URL = "/static/"
STATIC_ROOT = os.path.join(BASE_DIR, "static")
STATICFILES_DIRS = [os.path.join(BASE_DIR, "src/static_venv")]
\end{lstlisting}
\section{Cấu trúc giao diện Templates}
\subsection{Xử lý Frontend}
Phần này, chúng tôi sẽ trình bày về cách chúng tôi phân chia các file giao diện \textbf{\texttt{html}} trong thư mục \textbf{\texttt{templates}} được tạo ở hướng dẫn bên trên.
\par
Đầu tiên, chúng tôi tạo file \textbf{\texttt{base.html}} có chức năng là khung sườn cho toàn bộ giao diện với khả năng kết nạp các file khác để giúp chia nhỏ giao diện thành các phần có chức năng riêng biệt. Việc chia nhỏ giao diện thành các file thành phần giúp chúng tôi có thể quản lý code tốt hơn và tránh rối rắm khi nếu lưu quá nhiều dòng code trong một file duy nhất.
\par
Tiếp theo, chúng tôi tạo thêm hai file nữa có tên là \textbf{\texttt{header.html}} và \textbf{\texttt{footer.html}}. Quay lại file \textbf{\texttt{base.html}}, ta có cấu trúc code đơn giản như sau:
\begin{lstlisting}[language=html]
<!DOCTYPE html>
<html lang="vi">
<head>
    <title></title>
</head>
<body>
    <!-- Header -->
    
    <main>
        
    </main>
    <!-- Footer -->
    
    
</body>
</html>
\end{lstlisting}
\par
Để có thể thay đổi nội dung theo từng giao diện, chúng tôi đã sử dụng ba block là \textbf{\texttt{title}}, \textbf{\texttt{content}} và \textbf{\texttt{script}}, do đó chúng tôi sẽ thay đổi nội dung ở hai block này tùy theo mục đích mà chúng tôi muốn hướng đến.
\par
Ở file \textbf{\texttt{header.html}} và tương tự là file \textbf{\texttt{footer.html}} sẽ có nội dung cơ bản như sau:
\begin{lstlisting}[language=html]
<header>
    <nav>
        <a href="/"><h1>Danh Gia Web</h1></a>
    </nav>
</header>
\end{lstlisting}
\begin{lstlisting}[language=html]
<footer>
    <div>DGW &copy; 2018 - </div>
</footer>
\end{lstlisting}
\par
Sau khi cấu trúc xong bộ khung cho giao diện, tiếp theo chúng tôi sẽ xây dựng giao diện cho từng app dựa trên những gì đã thiết lập.
\par
Tại thư mục \textbf{\texttt{templates}} tạo thêm hai thư mục mới có tên trùng với hai app đã tạo là \textbf{\texttt{checkweb}} và \textbf{\texttt{tips}}. Chúng tôi sẽ đi sâu vào việc tạo giao diện cụ thể cho phần app \textbf{\texttt{checkweb}} vì tại đây là trọng tâm chính của ứng dụng đánh giá website của chúng tôi. Phần giao diện \textbf{\texttt{tips}} có phần đơn giản hơn nhiều và bạn có thể thiết lập dựa theo hướng dẫn ở phần \textbf{\texttt{checkweb}}. Hơn nữa, chúng tôi sẽ cung cấp mã nguồn ở phần \textbf{\textit{Kết luận}}, do đó bạn có thể tự nghiên cứu dựa theo những đoạn code của chúng tôi.
\par
Mở thư mục \textbf{\texttt{checkweb}} vừa tạo, dựa theo kiến trúc ở phần \textbf{\textit{Thiết kế giải pháp}}, chúng tôi tiến hành tạo thêm các file mới là \textbf{\texttt{index.html}}, \textbf{\texttt{about.html}}, \textbf{\texttt{contact.html}} và \textbf{\texttt{check.html}}.
\par
\textbf{\texttt{index.html}}
\begin{lstlisting}[language=html]

Trang Chu

<div>Noi dung Trang chu</div>

\end{lstlisting}
\par
Các file còn lại cũng có cấu trúc tương tự, với nội dung ở các block sẽ khác nhau tùy theo mỗi file. Ở đây chúng ta quan tâm đến hai file đó là \textbf{\texttt{index.html}} và \textbf{\texttt{check.html}} sẽ được nhắc đến ở những phần sau.
\subsection{Xử lý Backend}
Sau khi tạo xong các file \textbf{\texttt{html}}, tiếp theo chúng tôi sẽ tiến hành cấu hình để xử lý phần backend của ứng dụng.
\par
Đầu tiên, chúng tôi sẽ quản lý các url để hiển thị file giao diện trong ứng dụng tại file \textbf{\texttt{urls.py}} trong thư mục \textbf{\texttt{src}}. File \textbf{\texttt{urls.py}} sẽ có nội dung như sau:
\begin{lstlisting}[language=Python]
from django.urls import path, include

urlpatterns = [
    path("", include("checkweb.urls")),
    path("thu-thuat/", include("tips.urls")),
]
\end{lstlisting}
\par
Tiếp theo, chúng tôi thực hiện việc kết nối giữa truy vấn url và giao diện tại file \textbf{\texttt{views.py}} trong thư mục app \textbf{\texttt{checkweb}} được tạo ra khi chạy lệnh \textbf{\texttt{startapp}} lúc đầu. Django hỗ trợ việc kết nối này đơn giản và tiết kiệm dòng code hơn nhiều bằng chế độ \textbf{\textit{Class-based views}}. Chúng tôi sẽ sử dụng để tạo giao diện cho trang chủ, giới thiệu, liên hệ và trang kiểm tra.
\begin{lstlisting}[language=Python]
from django.views.generic import TemplateView

class IndexView(TemplateView):
    template_name = "checkweb/index.html"

class AboutView(TemplateView):
    template_name = "checkweb/about.html"

class ContactView(TemplateView):
    template_name = "checkweb/contact.html"

class CheckView(TemplateView):
    template_name = "checkweb/check.html"
\end{lstlisting}
\par
Trong thư mục \textbf{\texttt{checkweb}} hiện tại, tạo file \textbf{\texttt{urls.py}} để quản lý các url trong ứng dụng được gọi từ hàm \textbf{\texttt{include}} ở file \textbf{\texttt{urls.py}} trong thư mục \textbf{\texttt{src}}.
\begin{lstlisting}[language=Python]
from django.urls import path
from . import views

urlpatterns = [
    path("", views.IndexView.as_view(), name="index"),
    path("gioi-thieu/", views.AboutView.as_view(), name="about"),
    path("lien-he/", views.ContactView.as_view(), name="contact"),
    path("kiem-tra/", views.CheckView.as_view(), name="check"),
]
\end{lstlisting}
\par
Với cách tương tự, chúng tôi sẽ tiến hành cấu hình đối với app \textbf{\texttt{tips}}. Chi tiết, bạn có thể xem trên mã nguồn của chúng tôi.
\section{Hiện thực chức năng đánh giá website}
Sau khi thiết đặt xong phần giao diện templates, tiếp theo chúng tôi sẽ đi sâu vào xây dựng các hàm thực hiện chức năng chính cho project của mình. Đó là công việc nhận vào url trang web mà người dùng muốn đánh giá, kiểm tra captcha và url, sau đó trả về kết quả cho người dùng.
\par
Theo cách thông thường, các hàm này được viết trong file \textbf{\texttt{views.py}} ở thư mục app \textbf{\texttt{checkweb}}. Tuy nhiên, để dễ dàng quản lý hơn, chúng tôi sẽ tạo một file mới cùng trong thư mục app và đặt tên là \textbf{\texttt{functions.py}} sẽ chứa những đoạn code phục vụ cho việc phân tích và kiểm tra cho ứng dụng của chúng tôi. Do việc tách ra như vậy, tại file \textbf{\texttt{views.py}} chúng tôi sẽ nhúng file này bằng dòng code sau được thêm vào ở đầu file:
\begin{lstlisting}[language=Python]
from . import functions
\end{lstlisting}
\par
\subsection{Tạo form nhập và xử lý xác thực reCaptcha}
\subsubsection{Tạo giao diện nhập url}
Trước tiên, chúng tôi sẽ tạo giao diện form nhập để người dùng gửi url muốn kiểm tra vào ứng dụng của chúng tôi. Do đó, chúng tôi sẽ tiến hành thêm thẻ \textbf{\texttt{form}} vào file \textbf{\texttt{index.html}} trong thư mục \textbf{\texttt{templates/checkweb}}.
\begin{lstlisting}[language=html]
<form action="" method="POST">
    
    <input type="url" name="url" id="url" required>
    <button id="submit">Submit</button>
</form>
\end{lstlisting}
\par
Form của chúng tôi sẽ sử dụng giao thức \textbf{\texttt{POST}}, được xử lý trong backend và xuất về đường dẫn được khai báo trong \textbf{\texttt{urls.py}} có tên là \textbf{\texttt{check}}, cụ thể ở đây sẽ là file \textbf{\texttt{check.html}}.
\par
Sau khi đã tạo xong thẻ \textbf{\texttt{form}}, tiếp đến chúng tôi truy cập vào trang \url{https://www.google.com/recaptcha/admin/} để lấy thông tin về khóa API của reCaptcha. Khóa này được dùng để kích hoạt tính năng của reCaptcha. Theo hướng dẫn trên trang admin, chúng tôi sẽ chèn tiếp thẻ \textbf{\texttt{div}} bên trong thẻ \textbf{\texttt{form}} và bên dưới thẻ \textbf{\texttt{button}}.
\begin{lstlisting}[language=html]
<div class="g-recaptcha" data-sitekey="<SITE_KEY>" data-callback="onSubmit" data-badge="bottomleft" data-size="invisible"></div>
\end{lstlisting}
\par
Ngoài ra, để reCaptcha có thể hoạt động được, chúng tôi sẽ chèn thêm đoạn code để xử lý JavaScript. Đoạn code này chúng tôi sẽ đặt trong khối \textbf{\texttt{script}} được kế thừa từ khung \textbf{\texttt{base.html}} mà chúng tôi đã khai báo trước đó.
\par
Đoạn JavaScript bên dưới có nhiệm vụ bắt sự kiện click vào nút Summit và tạo ra đoạn mã xác thực reCaptcha mà chúng tôi sẽ xử lý tiếp ở phần tiếp theo:
\begin{lstlisting}[language=html]

<script src="https://www.google.com/recaptcha/api.js" async defer></script>
<script>
var sm = document.getElementById("submit");
sm.onclick = validate;

function validate(event) {
    event.preventDefault();
    grecaptcha.execute()
};

function onSubmit(token) {
    sm.onclick = null;
    sm.click();
}
</script>

\end{lstlisting}
\subsubsection{Hàm xác thực reCaptcha}
Chúng tôi sẽ viết hàm này trong file \textbf{\texttt{functions.py}} có tên là \textbf{\texttt{reCaptcha}}.
\begin{itemize}
    \item Đầu vào sẽ là mã được sinh ra từ việc xác thực của reCaptcha được lưu với tên là \textbf{\texttt{g-recaptcha-response}} từ truy vấn \textbf{\texttt{POST}} và IP của người dùng truy cập.
    \item Đầu ra là kết quả sau từ đánh giá của reCaptcha sau khi gửi các giá trị đầu vào lên máy chủ. Sẽ có hai giá trị là \textbf{\texttt{true}} và \textbf{\texttt{false}} tương ứng với kết quả xác thực người dùng là thành công hay thất bại.
\end{itemize}
\par
Khi đăng ký reCaptcha, chúng ta còn nhận được ngoài khóa \textbf{\texttt{SITE\_KEY}} thì còn một khóa nữa là \textbf{\texttt{SECRET\_KEY}}. Để có thể dễ dàng kiểm soát các khóa trong ứng dụng, chúng tôi sẽ lưu khóa bí mật này trong file \textbf{\texttt{settings.py}} ở thư mục \textbf{\texttt{src}}.
\begin{lstlisting}[language=Python]
# Google reCAPTCHA secret key
# https://developers.google.com/recaptcha/docs/verify/
    
GOOGLE_RECAPTCHA_SECRET_KEY = "<SECRET_KEY>"
\end{lstlisting}
\par
Và để sử dụng biến \textbf{\texttt{GOOGLE\_RECAPTCHA\_SECRET\_KEY}} thì Django có hỗ trợ bằng cách thêm dòng code sau vào đầu file \textbf{\texttt{functions.py}}:
\begin{lstlisting}[language=Python]
from django.conf import settings
\end{lstlisting}
\par
Bên cạnh đó, để gửi dữ liệu lên máy chủ của reCaptcha, chúng tôi cần thêm sự hỗ trợ của thư viện Python là \textbf{\texttt{requests}} mà chúng tôi đã cài đặt từ lúc mới thiết lập môi trường của ứng dụng.
\begin{lstlisting}[language=Python]
import requests
\end{lstlisting}
\par
Sau khi thêm các thư viện và ý tưởng xử lý thì hàm xử lý reCaptcha sẽ có nội dung như sau:
\begin{lstlisting}[language=Python]
data = {
    "secret": settings.GOOGLE_RECAPTCHA_SECRET_KEY,
    "response": response,
    "remoteip": userIP,
}
verify = requests.post(
    "https://www.google.com/recaptcha/api/siteverify", data=data)
result = verify.json()
return result["success"]
\end{lstlisting}
\par
Hàm \textbf{\texttt{reCaptcha}} được xây dựng xong thì tiếp theo, chúng tôi sẽ quay lại file \textbf{\texttt{views.py}} để gọi lại hàm này và xử lý nó.
\par
Việc xử lý captcha nằm trong class \textbf{\texttt{CheckView}} mà chúng tôi đã tạo trước đó. Theo quy ước trong Django thì để xử lý truy vấn \textbf{\texttt{POST}}, chúng tôi viết thêm hàm \textbf{\texttt{post}} trong class \textbf{\texttt{CheckView}}.
\begin{lstlisting}[language=Python]
def post(self, request):
    url = request.POST["url"]
    if reCaptcha(request.POST["g-recaptcha-response"], request.META["REMOTE_ADDR"]):
        # code check web here
        return render(request, "checkweb/check.html")
    return redirect("/")
\end{lstlisting}
\par
Nếu kiểm tra captcha thành công thì ứng dụng sẽ trả về trang giao diện trong file \textbf{\texttt{check.html}}. Nếu xác thực thất bại thì sẽ chuyển hướng trở lại trang chủ. Chúng tôi có sử dụng hàm chuyển hướng trang \textbf{\texttt{redirect}}. Thêm vào đầu file \textbf{\texttt{views.py}} đoạn code sau để chèn hàm này:
\begin{lstlisting}[language=Python]
from django.shortcuts import redirect
\end{lstlisting}
\par
Đến đây, chúng tôi đã giải quyết xong vấn đề kiểm tra xác thực người dủng bằng reCaptcha.
\subsection{Phân tích cấu trúc nội dung website}
\subsubsection{Lấy source code và phân tách trang web}
Theo cách hoạt động ứng dụng của chúng tôi sẽ phân tích trang web bằng cách lấy source code tương đương với tính năng View source trên các trình duyệt web. Để xử lý chức năng này, chúng tôi xây dựng hàm \textbf{\texttt{parsing}} trong file \textbf{\texttt{functions.py}}.
\begin{itemize}
    \item Đầu vào sẽ là url của trang web cần kiểm tra
    \item Đầu ra là biến có kiểu \textbf{\texttt{dict()}} chứa nội dung các yếu tố mà chúng tôi phân tách ra được từ source code.
\end{itemize}
\par
Ngoài thư viện \textbf{\texttt{requests}}, chúng tôi cần sự trợ giúp thêm của thư viện \textbf{\texttt{lxml}}.
\begin{lstlisting}[language=Python]
from lxml import html
\end{lstlisting}
\par
Sau khi thêm vào thư viện cần thiết, bên dưới là đoạn code dùng để lấy source code dựa trên url trang web.
\begin{lstlisting}[language=Python]
def parsing(url):
    try:
        page = requests.get(url, timeout=5)
        content = html.fromstring(page.content.decode("utf-8"))
    except BaseException:
        return False
\end{lstlisting}
\par
Vì để tránh chương trình bị lỗi khi người dùng nhập vào url mà chúng tôi không thể kiểm tra được nên chúng tôi đã sử dụng cấu trúc \textbf{\texttt{try/except}} trong Python để giải quyết vấn đề này. Ngoài ra, nếu truy vấn vượt quá năm giây cũng sẽ bị xem là lỗi không kiểm tra được. Cuối cùng, biến \textbf{\texttt{content}} được decode theo chuẩn mã \textbf{\texttt{utf-8}} để hiển thị font chữ không bị lỗi.
\par
Tiếp theo, cũng trong hàm \textbf{\texttt{parsing}}, chúng tôi khai báo biến \textbf{\texttt{value}} có kiểu là \textbf{\texttt{dict()}} để lưu các yếu tố được phân tách. Bên cạnh đó là các biến dùng để phục vụ làm input cho các hàm về sau.
\begin{lstlisting}[language=Python]
value = dict()
domain = url.split("/")[2]
urlDomain = url.split("/")[0] + "//" + domain    
\end{lstlisting}
\par
Việc phân tách cấu trúc trong source code, chúng tôi sử dụng phương pháp \textbf{\texttt{xpath}}. Để dễ dàng quản lý, chúng tôi chia ra hai loại là thành phần chỉ có một giá trị, ví dụ thẻ \textbf{\texttt{title}} sẽ được lưu cấu trúc \textbf{\texttt{xpath}} trong biến \textbf{\texttt{elm}}, những thành phần có hơn một giá trị, ví dụ thẻ \textbf{\texttt{img}} sẽ được lưu cấu trúc \textbf{\texttt{xpath}} trong biến \textbf{\texttt{elms}}. Cả hai biến \textbf{\texttt{elm}} và \textbf{\texttt{elms}} đều cò kiểu là \textbf{\texttt{dict()}}.
\begin{lstlisting}[language=Python]
elm = {
    "title": "//title/text()",
    "description": "//meta[@name=`description']/@content",
    "favicon": "//link[contains(@rel, `icon')]/@href",
    "robotsMeta": "//meta[@name=`robots']/@content",
}
elms = {
    "h1Tags": "//h1//text()",
    "h2Tags": "//h2//text()",
    "aTags": "//a/@href",
    "cssInlines": "//@style/..",
    "imgTags": "//img",
}
\end{lstlisting}
\par
Sau khi đã có cấu trúc \textbf{\texttt{xpath}} của từng thuộc tính, tiếp theo chúng tôi sẽ sử dụng vòng lặp \textbf{\texttt{for}} để lấy dữ liệu dựa trên cấu trúc có sẵn. Do có hai cấu trúc khác khau, nên chúng tôi cũng sẽ dùng hai vòng lặp để xử lý.
\begin{lstlisting}[language=Python]
for k, v in elm.items():
    try:
        value[k] = content.xpath(v)[0]
    except BaseException:
        value[k] = None

for k, v in elms.items():
    try:
        value[k] = content.xpath(v)
    except BaseException:
        value[k] = None
\end{lstlisting}
\par
Sau khi chạy xong hai vòng lặp này thì biến \textbf{\texttt{value}} đã lưu được khá nhiều dữ liệu từ việc phân tách source code dựa trên \textbf{\texttt{xpath}}. Tuy nhiên, có một số yếu tố sẽ không đúng chuẩn do những website khác nhau dẫn đến có nhiều kết quả không như ý. chúng tôi sẽ giải quyết vấn đề này bằng các hàm bổ sung và được chúng tôi đề cập trong những phần tiếp theo.
\par
Cuối cùng, chúng tôi trả về kết quả của biến \textbf{\texttt{value}} và chuyển qua file \textbf{\texttt{views.py}} để xử lý dữ liệu trả về.
\begin{lstlisting}[language=Python]
return value
\end{lstlisting}
\par
Tiếp theo, chúng tôi sẽ xử lý dữ liệu của \textbf{\texttt{value}} ở file \textbf{\texttt{views.py}}. Đoạn code bên dưới bao gồm luôn phần xử lý xác thực reCaptcha được chúng tôi trình bày ở phần trước.
\begin{lstlisting}[language=Python]
def post(self, request):
    url = request.POST["url"]
    if reCaptcha(request.POST["g-recaptcha-response"], request.META["REMOTE_ADDR"]):
        context = parsing(url)
        if context:
            context["url"] = url
            return render(request, "checkweb/check.html", context)
        return redirect("/")
    return redirect("/")
\end{lstlisting}
\par
Biến \textbf{\texttt{context}} sẽ nhận kết quả trả về từ hàm \textbf{parsing}. Nếu không có giá trị, nghĩa là việc kiểm tra url gặp lỗi, do đó chúng tôi thay vì trả về truy vấn đến trang giao diện \textbf{\texttt{check.html}}, mà sẽ chuyển hướng trở về trang chủ.
\subsection{Xử lý nâng cao các cấu trúc trong website}
\subsubsection{Favicon}
Vấn đề xảy ra ở đây là do định dạng cấu trúc đường dẫn đến hình ảnh của các website khác nhau có phần khác biệt, và do \textbf{\texttt{xpath}} chỉ lấy nội dung bên trong thẻ nên chúng tôi cần phải định dạng lại cấu trúc đường dẫn này. Ví dụ, liên kết hình ảnh bắt đầu bằng \textbf{\texttt{`/'}}, để hiển thị đúng thì cần phải gắn thêm phần tên miền vào trước liên kết. Sau đây là hàm xử lý trong file \textbf{\texttt{functions.py}}
\begin{lstlisting}[language=Python]
def getLinkImg(elm, urlDomain):
    if elm and elm[:2] not in {"ht", "//"}:
        elm = urlDomain + "/" + elm.lstrip("/")
    return elm
\end{lstlisting}
\par
Chúng ta sẽ gán lại khóa này trong biến \textbf{\texttt{value}}:
\begin{lstlisting}[language=Python]
value["favicon"] = getLinkImg(value["favicon"], urlDomain)
\end{lstlisting}
\subsubsection{Thẻ H1, H2}
Trong quá trình phát triển, chúng tôi nhận thấy nội dung của các thẻ này thường xuyên bị thừa các ký tự khoảng trắng ở đầu hoặc cuối dòng. Đôi khi có vài thẻ có nội dung rỗng. Nguyên nhân là vì bên trong các thẻ này còn được lồng thêm các thẻ khác như thẻ \textbf{\texttt{a}}, \textbf{\texttt{span}}. Do đó chúng tôi cần viết thêm hàm để dọn sạch các khoảng trống bị dư thừa, cũng như các thẻ không chứa nội dung.
\begin{lstlisting}[language=Python]
def cleanElms(elms):
    if elms:
        for idx, _ in enumerate(elms):
            elms[idx] = elms[idx].strip()
        elms = list(filter(None, elms))
    return elms
\end{lstlisting}
\par
Sau đó, chúng tôi gán lại dữ liệu cho các thẻ này trong hàm \textbf{\texttt{parsing}}:
\begin{lstlisting}[language=Python]
value["h1Tags"] = cleanElms(value["h1Tags"])
value["h2Tags"] = cleanElms(value["h2Tags"])
\end{lstlisting}
\subsubsection{Robots.txt}
Mặc định, nếu trang web sử dụng file này thì sẽ có cấu trúc đường dẫn là \url{https://ten-mien/robots.txt}. Do đó, chúng tôi sẽ sử dụng thư viện \textbf{\texttt{requests}} để truy vấn đến liên kết này và kiểm tra xem mã trạng thái trả về có phải là \textbf{\texttt{200}} không, cũng như có kiểu nội dung là \textbf{\texttt{plain}}.
\begin{lstlisting}[language=Python]
def getlinkRobots(urlDomain):
    try:
        value = requests.get(urlDomain + "/robots.txt")
    except BaseException:
        return None
    if value.status_code != 200 or "plain" not in value.headers["Content-Type"]:
        return None
    return urlDomain + "/robots.txt"
\end{lstlisting}
\par
Đây là yếu tố mới không có trong biến \textbf{\texttt{value}} khi xử lý \textbf{\texttt{xpath}}, do đó chúng tôi gán giá trị hàm này vào khóa mới:
\begin{lstlisting}[language=Python]
value["robotsTxt"] = getlinkRobots(urlDomain)
\end{lstlisting}
\subsubsection{Sitemap.xml}
Thông thường, cũng tương tự như với \textbf{\texttt{robots.txt}}, đường dẫn của \textbf{\texttt{sitemap.xml}} sẽ là \url{https://ten-mien/sitemap.xml}. Tuy nhiên, ở một số các trang web, liên kết \textbf{\texttt{sitemap.xml}} không giống như mặc định và đường dẫn đó được khai báo trong file \textbf{\texttt{robots.txt}} có nội dung tương tự như:
\begin{lstlisting}
Sitemap: https://ten-mien/site-map/sitemap.xml
\end{lstlisting}
\par
Do đó, để kiểm tra đường dẫn \textbf{\texttt{sitemap.xml}} của trang web, chúng tôi sẽ xem xét nội dung trong file \textbf{\texttt{robots.txt}} trước, nếu không tìm thấy, chúng tôi sẽ tiếp tục kiểm tra với liên kết mặc định. Đối với mỗi liên kết có được, chúng tôi sẽ kiểm tra mã trạng thái trả về có là \textbf{\texttt{200}} hay không, cũng như kiểu nội dung là \textbf{\texttt{xml}}.
\begin{lstlisting}[language=Python]
def getlinkSitemap(urlDomain, robots):
    try:
        value = requests.get(urlDomain + "/sitemap.xml")
    except BaseException:
        return None
    if robots:
        txt = requests.get(robots).content.decode("utf-8")
        txt = txt.replace("\n", "")
        sitemap = re.findall(r"Sitemap:.*xml", txt)
        if sitemap:
            sitemap = sitemap[0].split("Sitemap: ")[1:]
            return sitemap
    if value.status_code != 200 or "xml" not in value.headers["Content-Type"]:
        return None
    sitemap = [urlDomain + "/sitemap.xml"]
    return sitemap
\end{lstlisting}
\par
Cũng tương tự yếu tố \textbf{\texttt{robots.txt}}, chúng tôi tiếp tục thêm khóa mới vào biến \textbf{\texttt{value}}:
\begin{lstlisting}[language=Python]
value["sitemaps"] = getlinkSitemap(urlDomain, value["robotsTxt"])
\end{lstlisting}